%\documentclass[reprint,superscriptaddress]{revtex4-1}
\documentclass[aip,jcp,preprint,superscriptaddress]{revtex4-1}
\usepackage{amsmath}
\usepackage{tikz}
\usepackage{color}
\begin{document}




\newcommand{\vct}[1]{\mathbf{#1}}
\newcommand{\vx}{\vct{x}}
\newcommand{\Z}{\mathcal{Z}}
\newcommand{\E}{\mathcal{E}}




\title{Notes}

\maketitle



\section{Mnemonic derivation of WHAM}



We give a mnemonic derivation of the expression
for the density of states
%
\begin{equation}
g(E)
=
\frac{
  \sum_{k = 1}^K n_k(E)
}
{
  \sum_{k = 1}^K N_k \, \exp(-\beta_k E) / Z_k
}.
\label{eq:gE_WHAM}
\end{equation}
%

In ensemble $k$,
the normalized energy distribution,
$n_k(E) / N_k$,
is related to the density of states as
%
\begin{equation}
\frac{ n_k(E) } { N_k }
=
g(E) \,
\frac{ \exp(-\beta_k E) } { Z_k }.
\end{equation}
%
Thus,
we have
%
\begin{equation}
g(E)
= \frac{ n_k(E) }
{ N_k \, \exp(-\beta_k E) / Z_k }.
\end{equation}
%
Taking the harmonic average over $k$
yields Eq. \eqref{eq:gE_WHAM}.




\section{Parameters for Figure 1}



The illustration was produced according to the following potential
%
\begin{equation}
F
=
\frac{1}{2} A \, ( x - x_0 )^2
+
\frac{1}{2} B \, ( y - y_0 )^2
+
C \, ( x - x_0 ) ( y - y_0 ),
\label{eq:fig1_F}
\end{equation}
%
where
$x_0 = 3/5$,
$y_0 = 4/5$,
$A = 3/4$,
$B = 5/4$,
and
$C = 1/4$.



The two vectors are
$\vct f_1 = (1, 0)$
and
$\vct f_2 = (0, 1)$.
%
The residual vectors are
$\vct R_1 = (-1/10, 9/10)$
and
$\vct R_2 = (4/10, -1/10)$,
respectively.



We now determine
$\vct{\hat R} = \vct R_1 + \lambda \, (\vct R_2 - \vct R_1)$
with the minimal magnitude
$\| \vct{\hat R} \|$.
%
It is readily shown that
$\lambda$ satisfies
%
\[
\lambda
=
-\frac{
  \Delta \vct R \cdot \vct R_1
}
{
  \Delta \vct R \cdot \Delta \vct R
}
=
-\frac{-19/20}{5/4}
=
\frac{19}{25},
\]
%
where
$\Delta \vct R = \vct R_2 - \vct R_1 = (1/2, -1)$.
%
Thus,
\[
\vct{\hat R}
=
(1 - \lambda) \, \vct R_1
+ \lambda \, \vct R_2
=
\left(
  \frac{7}{25},
  \frac{7}{50}
\right),
\]
\[
\vct{\hat f}
=
(1 - \lambda) \, \vct f_1
+ \lambda \, \vct f_2
=
\left(
  \frac{6}{25},
  \frac{19}{25}
\right),
\]
and
\[
\vct{\hat f}
+
\vct{\hat R}
=
\left(
  \frac{13}{25},
  \frac{9}{10}
\right).
\]



We can complete squares of $F$ as
%
\begin{align*}
F
&=
\frac{1}{16}
\left\{
  \left( 5 + 3 \sqrt 2 \right)
  \left[
    \left( \sqrt 2 - 1 \right) \, \bar x
  +
    \bar y
  \right]^2
  +
  \left( 3 + \sqrt 2 \right)
  \left[
    \bar x
  -
    \left( \sqrt 2 - 1 \right) \, \bar y
  \right]^2
\right\}
\notag \\
%
&=
\frac{1}{16}
\left\{
  \left( 3 - \sqrt 2 \right)
  \left[
    \bar x
  +
    \left( \sqrt 2 + 1 \right) \, \bar y
  \right]^2
  +
  \left( 3 + \sqrt 2 \right)
  \left[
    \bar x
  -
    \left( \sqrt 2 - 1 \right) \, \bar y
  \right]^2
\right\}
\notag \\
%
&=
\frac{7}{2} R^2,
\end{align*}
%
where
%
\begin{align*}
\bar x
&= x - x_0
\\
&=
\left(
  \sqrt{ 5 - 3 \sqrt{2} } \, \cos \theta
  +
  \sqrt{ 5 + 3 \sqrt{2} } \, \sin \theta
\right) R
\\
&=
\left[
  \left( \sqrt 2 - 1 \right) \sqrt{ 3 + \sqrt{2} } \, \cos \theta
  +
  \left( \sqrt 2 + 1 \right) \sqrt{ 3 - \sqrt{2} } \, \sin \theta
\right] R,
\end{align*}
%
and
%
\begin{align*}
\bar y
&= y - y_0
\\
&=
\left(
  \sqrt{ 3 + \sqrt{2} } \, \cos \theta
  -
  \sqrt{ 3 - \sqrt{2} } \, \sin \theta
\right) R,
\end{align*}



The short axis is achieved at $\theta = 0$,
with
$\sqrt{ {\bar x}^2 + {\bar y}^2 } = \sqrt{ 8 - \sqrt 8 } R$,
and
$\phi = \arctan(\bar y/\bar x) = \arctan \left(\sqrt 2 - 1\right) = \pi/8$.



The long axis is achieved at $\theta = \pi/2$,
with
$\sqrt{ {\bar x}^2 + {\bar y}^2 } = \sqrt{ 8 + \sqrt 8 } R$,
and
$\phi = \arctan(\bar y/\bar x) = \arctan \left(-\sqrt 2 - 1\right) = -3 \, \pi/8$.





%\bibliography{simul}
\end{document}
