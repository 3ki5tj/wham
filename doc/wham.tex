\documentclass[preprint]{revtex4-1}
\usepackage{amsmath}
\begin{document}




\newcommand{\vct}[1]{\mathbf{#1}}



\section{Introduction}



The multiple histogram method\cite{
  ferrenberg1988, *ferrenberg1989}
or its generalization,
the weighted histogram analysis method (WHAM)\cite{
  kumar1992, souaille2001, kastner2005,
  chodera2007, bereau2009,
  kim2011},
is a popular method of estimating
the free energy difference
between thermodynamic states
of different conditions (e.g. temperature and pressure).
%
The method iteratively solves a pair of coupled equations of
the free energy and the density of states
involving the energy histograms, hence the name.
%
Although the original formulation is
based on histograms,
this restriction can be lifted,
resulting in the
multistate Bennett acceptance ratio (MBAR) method\cite{
shirts2008},
which direct extends the original
Bennett acceptance ratio (BAR) method\cite{
bennett1976}.



Although powerful,
the standard implementation of WHAM or MBAR can suffer from
a slow convergence in later stages.
%
Several remedies are proposed\cite{
shirts2008, bereau2009, kim2011}.
%
For example, one may use the Newton-Raphson method
to directly compute the Hessian matrix\cite{
shirts2008}, although it can be occasionally unstable.
%
An ingenious non-iterative alternative
is to directly estimate the derivative of density of states,
leading to the statistical-temperature WHAM\cite{
kim2011}.
%
This variant, however, gives
a slightly different free energy from the original WHAM,
and the extension to multiple-dimensional,
e.g., isothermal-isobaric, ensembles
is not obvious.



Here, we consider improving WHAM using
the method of direct inversion in the iterative subspace (DIIS)\cite{
pulay1980, *pulay1982, *hamilton1986}.
%
Particularly, we consider a modification
popular in the liquid state theory\cite
{kovalenko1999, howard2011}.




\section{Method}



\subsection{WHAM}



The mutilple histogram method or WHAM,



\subsection{MDIIS}


The mixing factor is $1.0$ in this study.


The basis is then updated by the new vector.
%
In a popular updating scheme\cite{kovalenko1999},
we add the new vector to the basis,
if there is fewer than $M$ vectors in the basis,
%
or replace the earliest vector by the new vector.
%
If, however, the new vector
produces an error greater than
(in terms of the norm of residual vector)
$K_r$ (here, $K_r = 10.0$ as recommended)
times the minimal error of the basis,
%
we rebuild the basis
from the least erroneous vector in basis.



An alternative is the following.
%
First, we find the most erroneous vector,
$\vct u_m$, from the basis.
%and compute its error, $\varepsilon(\vct u_m)$.
%
If the new vector, $\vct v$,
produces an error less than $\vct u_m$,
we add $\vct v$ into the basis
(and remove $\vct u_m$ if the basis is full).
%
Otherwise,
we remove $\vct u_m$ from the basis
and if this leaves the basis empty,
we rebuild the basis from $\vct v$.
%
% We find the modification is slightly better.



\section{Results}



The method is tested on three systems.


The first is the two-dimensional $64\times64$ Ising model.


Lennard-Jones, canonoical ($NVT$), isothermal-isobaric ($NPT$) ensemble.



Molecular system of villin headpiece,
The protein is immersed in
a dodecahedron box with 1898 water molecules and two chloride ions.
%
Both $NVT$ and $NPT$ ensembles are used.





\section{Acknowledgements}



It is a pleasure to thank Dr. Y. Zhao
for many helpful discussions.
%
Computer time on the Lonestar supercomputer
at the Texas Advanced Computing Center
at the University of Texas at Austin
is gratefully acknowledged.



\bibliography{simul}
\end{document}
